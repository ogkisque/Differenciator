\documentclass[a4paper,12pt]{article}
\usepackage[T1]{fontenc}
\usepackage[utf8]{inputenc}
\usepackage[english,russian]{babel}
\usepackage{pdfpages}
\usepackage{amsmath,amsfonts,amssymb,amsthm,mathtools}
\usepackage[left=10mm, top=10mm, right=10mm, bottom=20mm, nohead, nofoot]{geometry}
\usepackage{wasysym}
\author{\LARGEМерзляков Арсений}
\title{Анализ функции}
\pagestyle {empty}
\begin{document}
\maketitle
\begin{flushleft}
\Large
$f(x, y, z) = ((-789)+\cos {(((-1) \cdot x+(y)^{z}))}) \cdot (-1) \cdot (\ln {((x \cdot 7-y))})^{(-1) \cdot z}$

Выполним самую простую вещь, которую я встречал за 18 лет жизни - расчёт частных производных:

\newpageДифференцируем по x:

$g(x) = y$

Если приглядеться, то ты все равно не увидишь, что:

$g^{'}(x) = 0$

$g(x) = 7$

Сколько бы я не терял память, никогда не забуду, что:

$g^{'}(x) = 0$

$g(x) = x$

Я бы перерезал себе вены, если бы не знал, что:

$g^{'}(x) = 1$

$g(x) = x \cdot 7$

Я сам выпал на этом моменте:

$g^{'}(x) = (1 \cdot 7+x \cdot 0)$

$g(x) = (x \cdot 7-y)$

Сколько бы я не терял память, никогда не забуду, что:

$g^{'}(x) = ((1 \cdot 7+x \cdot 0)-0)$

$g(x) = \ln {((x \cdot 7-y))}$

Петрович закопает того, кто не знает, что:

$g^{'}(x) =  \dfrac{1}{(x \cdot 7-y)}  \cdot ((1 \cdot 7+x \cdot 0)-0)$

$g(x) = \ln {(\ln {((x \cdot 7-y))})}$

Ллойд и Гарри из 'Тупой и ещё тупее' знали, что:

$g^{'}(x) =  \dfrac{1}{\ln {((x \cdot 7-y))}}  \cdot  \dfrac{1}{(x \cdot 7-y)}  \cdot ((1 \cdot 7+x \cdot 0)-0)$

$g(x) = z$

Максимально тривиально, что:

$g^{'}(x) = 0$

$g(x) = (-1)$

Я бы перерезал себе вены, если бы не знал, что:

$g^{'}(x) = 0$

$g(x) = (-1) \cdot z$

Заметим, что:

$g^{'}(x) = (0 \cdot z+(-1) \cdot 0)$

$g(x) = (-1) \cdot z \cdot \ln {(\ln {((x \cdot 7-y))})}$

Сколько бы я не терял память, никогда не забуду, что:

$g^{'}(x) = ((0 \cdot z+(-1) \cdot 0) \cdot \ln {(\ln {((x \cdot 7-y))})}+(-1) \cdot z \cdot  \dfrac{1}{\ln {((x \cdot 7-y))}}  \cdot  \dfrac{1}{(x \cdot 7-y)}  \cdot ((1 \cdot 7+x \cdot 0)-0))$

$g(x) = (\ln {((x \cdot 7-y))})^{(-1) \cdot z}$

Ежу понятно, что:

$g^{'}(x) = (\ln {((x \cdot 7-y))})^{(-1) \cdot z} \cdot ((0 \cdot z+(-1) \cdot 0) \cdot \ln {(\ln {((x \cdot 7-y))})}+(-1) \cdot z \cdot  \dfrac{1}{\ln {((x \cdot 7-y))}}  \cdot  \dfrac{1}{(x \cdot 7-y)}  \cdot ((1 \cdot 7+x \cdot 0)-0))$

$g(x) = (-1)$

Если бы меня разбудили в ночь после посвята, то я бы сходу ответил, что:

$g^{'}(x) = 0$

$g(x) = (-1) \cdot (\ln {((x \cdot 7-y))})^{(-1) \cdot z}$

Ньютон перевернулся бы в гробу, если бы узнал, что ты не знаешь, что:

$g^{'}(x) = (0 \cdot (\ln {((x \cdot 7-y))})^{(-1) \cdot z}+(-1) \cdot (\ln {((x \cdot 7-y))})^{(-1) \cdot z} \cdot ((0 \cdot z+(-1) \cdot 0) \cdot \ln {(\ln {((x \cdot 7-y))})}+(-1) \cdot z \cdot  \dfrac{1}{\ln {((x \cdot 7-y))}}  \cdot  \dfrac{1}{(x \cdot 7-y)}  \cdot ((1 \cdot 7+x \cdot 0)-0)))$

$g(x) = z$

Если бы меня разбудили в ночь после посвята, то я бы сходу ответил, что:

$g^{'}(x) = 0$

$g(x) = (y)^{z}$

Если приглядеться, то ты все равно не увидишь, что:

$g^{'}(x) = \ln {(y)} \cdot (y)^{z} \cdot 0$

$g(x) = x$

Если бы меня разбудили в ночь после посвята, то я бы сходу ответил, что:

$g^{'}(x) = 1$

$g(x) = (-1)$

Даже ИНБИКСТ знает, что:

$g^{'}(x) = 0$

$g(x) = (-1) \cdot x$

Не понимаю тех, кто не знает, что:

$g^{'}(x) = (0 \cdot x+(-1) \cdot 1)$

$g(x) = ((-1) \cdot x+(y)^{z})$

Максимально тривиально, что:

$g^{'}(x) = ((0 \cdot x+(-1) \cdot 1)+\ln {(y)} \cdot (y)^{z} \cdot 0)$

$g(x) = \cos {(((-1) \cdot x+(y)^{z}))}$

Если бы моя собака умела говорить, то сказала бы, что:

$g^{'}(x) = \sin {(((-1) \cdot x+(y)^{z}))} \cdot (-1) \cdot ((0 \cdot x+(-1) \cdot 1)+\ln {(y)} \cdot (y)^{z} \cdot 0)$

$g(x) = (-789)$

В садике мне рассказывали, что:

$g^{'}(x) = 0$

$g(x) = ((-789)+\cos {(((-1) \cdot x+(y)^{z}))})$

Сложить 2 + 2 эквивалентно по сложности следующему:

$g^{'}(x) = (0+\sin {(((-1) \cdot x+(y)^{z}))} \cdot (-1) \cdot ((0 \cdot x+(-1) \cdot 1)+\ln {(y)} \cdot (y)^{z} \cdot 0))$

$g(x) = ((-789)+\cos {(((-1) \cdot x+(y)^{z}))}) \cdot (-1) \cdot (\ln {((x \cdot 7-y))})^{(-1) \cdot z}$

Если бы моя собака умела говорить, то сказала бы, что:

$g^{'}(x) = ((0+\sin {(((-1) \cdot x+(y)^{z}))} \cdot (-1) \cdot ((0 \cdot x+(-1) \cdot 1)+\ln {(y)} \cdot (y)^{z} \cdot 0)) \cdot (-1) \cdot (\ln {((x \cdot 7-y))})^{(-1) \cdot z}+((-789)+\cos {(((-1) \cdot x+(y)^{z}))}) \cdot (0 \cdot (\ln {((x \cdot 7-y))})^{(-1) \cdot z}+(-1) \cdot (\ln {((x \cdot 7-y))})^{(-1) \cdot z} \cdot ((0 \cdot z+(-1) \cdot 0) \cdot \ln {(\ln {((x \cdot 7-y))})}+(-1) \cdot z \cdot  \dfrac{1}{\ln {((x \cdot 7-y))}}  \cdot  \dfrac{1}{(x \cdot 7-y)}  \cdot ((1 \cdot 7+x \cdot 0)-0))))$

После очевиднейших упрощений, которые адекватный человек может сделать ещё в утробе, получаем:

$(\sin {(((-1) \cdot x+(y)^{z}))} \cdot (-1) \cdot (-1) \cdot (-1) \cdot (\ln {((x \cdot 7-y))})^{(-1) \cdot z}+((-789)+\cos {(((-1) \cdot x+(y)^{z}))}) \cdot (-1) \cdot (\ln {((x \cdot 7-y))})^{(-1) \cdot z} \cdot (-1) \cdot z \cdot  \dfrac{1}{\ln {((x \cdot 7-y))}}  \cdot  \dfrac{1}{(x \cdot 7-y)}  \cdot 7)$

\newpageДифференцируем по y:

$g(y) = y$

Я сам выпал на этом моменте:

$g^{'}(y) = 1$

$g(y) = 7$

Я бы перерезал себе вены, если бы не знал, что:

$g^{'}(y) = 0$

$g(y) = x$

Ньютон перевернулся бы в гробу, если бы узнал, что ты не знаешь, что:

$g^{'}(y) = 0$

$g(y) = x \cdot 7$

Сложить 2 + 2 эквивалентно по сложности следующему:

$g^{'}(y) = (0 \cdot 7+x \cdot 0)$

$g(y) = (x \cdot 7-y)$

Даже ИНБИКСТ знает, что:

$g^{'}(y) = ((0 \cdot 7+x \cdot 0)-1)$

$g(y) = \ln {((x \cdot 7-y))}$

Если приглядеться, то ты все равно не увидишь, что:

$g^{'}(y) =  \dfrac{1}{(x \cdot 7-y)}  \cdot ((0 \cdot 7+x \cdot 0)-1)$

$g(y) = \ln {(\ln {((x \cdot 7-y))})}$

Ежу понятно, что:

$g^{'}(y) =  \dfrac{1}{\ln {((x \cdot 7-y))}}  \cdot  \dfrac{1}{(x \cdot 7-y)}  \cdot ((0 \cdot 7+x \cdot 0)-1)$

$g(y) = z$

Максимально тривиально, что:

$g^{'}(y) = 0$

$g(y) = (-1)$

Ежу понятно, что:

$g^{'}(y) = 0$

$g(y) = (-1) \cdot z$

Максимально тривиально, что:

$g^{'}(y) = (0 \cdot z+(-1) \cdot 0)$

$g(y) = (-1) \cdot z \cdot \ln {(\ln {((x \cdot 7-y))})}$

Ньютон перевернулся бы в гробу, если бы узнал, что ты не знаешь, что:

$g^{'}(y) = ((0 \cdot z+(-1) \cdot 0) \cdot \ln {(\ln {((x \cdot 7-y))})}+(-1) \cdot z \cdot  \dfrac{1}{\ln {((x \cdot 7-y))}}  \cdot  \dfrac{1}{(x \cdot 7-y)}  \cdot ((0 \cdot 7+x \cdot 0)-1))$

$g(y) = (\ln {((x \cdot 7-y))})^{(-1) \cdot z}$

Я сам выпал на этом моменте:

$g^{'}(y) = (\ln {((x \cdot 7-y))})^{(-1) \cdot z} \cdot ((0 \cdot z+(-1) \cdot 0) \cdot \ln {(\ln {((x \cdot 7-y))})}+(-1) \cdot z \cdot  \dfrac{1}{\ln {((x \cdot 7-y))}}  \cdot  \dfrac{1}{(x \cdot 7-y)}  \cdot ((0 \cdot 7+x \cdot 0)-1))$

$g(y) = (-1)$

Сколько бы я не терял память, никогда не забуду, что:

$g^{'}(y) = 0$

$g(y) = (-1) \cdot (\ln {((x \cdot 7-y))})^{(-1) \cdot z}$

Если спросить у рандомного бомжа на улице, то он будет знать, что:

$g^{'}(y) = (0 \cdot (\ln {((x \cdot 7-y))})^{(-1) \cdot z}+(-1) \cdot (\ln {((x \cdot 7-y))})^{(-1) \cdot z} \cdot ((0 \cdot z+(-1) \cdot 0) \cdot \ln {(\ln {((x \cdot 7-y))})}+(-1) \cdot z \cdot  \dfrac{1}{\ln {((x \cdot 7-y))}}  \cdot  \dfrac{1}{(x \cdot 7-y)}  \cdot ((0 \cdot 7+x \cdot 0)-1)))$

$g(y) = y$

Если бы меня разбудили в ночь после посвята, то я бы сходу ответил, что:

$g^{'}(y) = 1$

$g(y) = (y)^{z}$

Ллойд и Гарри из 'Тупой и ещё тупее' знали, что:

$g^{'}(y) = z \cdot (y)^{(z-1)} \cdot 1$

$g(y) = x$

Ежу понятно, что:

$g^{'}(y) = 0$

$g(y) = (-1)$

Ньютон перевернулся бы в гробу, если бы узнал, что ты не знаешь, что:

$g^{'}(y) = 0$

$g(y) = (-1) \cdot x$

Заметим, что:

$g^{'}(y) = (0 \cdot x+(-1) \cdot 0)$

$g(y) = ((-1) \cdot x+(y)^{z})$

Не понимаю тех, кто не знает, что:

$g^{'}(y) = ((0 \cdot x+(-1) \cdot 0)+z \cdot (y)^{(z-1)} \cdot 1)$

$g(y) = \cos {(((-1) \cdot x+(y)^{z}))}$

Ньютон перевернулся бы в гробу, если бы узнал, что ты не знаешь, что:

$g^{'}(y) = \sin {(((-1) \cdot x+(y)^{z}))} \cdot (-1) \cdot ((0 \cdot x+(-1) \cdot 0)+z \cdot (y)^{(z-1)} \cdot 1)$

$g(y) = (-789)$

Заметим, что:

$g^{'}(y) = 0$

$g(y) = ((-789)+\cos {(((-1) \cdot x+(y)^{z}))})$

Не понимаю тех, кто не знает, что:

$g^{'}(y) = (0+\sin {(((-1) \cdot x+(y)^{z}))} \cdot (-1) \cdot ((0 \cdot x+(-1) \cdot 0)+z \cdot (y)^{(z-1)} \cdot 1))$

$g(y) = ((-789)+\cos {(((-1) \cdot x+(y)^{z}))}) \cdot (-1) \cdot (\ln {((x \cdot 7-y))})^{(-1) \cdot z}$

Если бы меня разбудили в ночь после посвята, то я бы сходу ответил, что:

$g^{'}(y) = ((0+\sin {(((-1) \cdot x+(y)^{z}))} \cdot (-1) \cdot ((0 \cdot x+(-1) \cdot 0)+z \cdot (y)^{(z-1)} \cdot 1)) \cdot (-1) \cdot (\ln {((x \cdot 7-y))})^{(-1) \cdot z}+((-789)+\cos {(((-1) \cdot x+(y)^{z}))}) \cdot (0 \cdot (\ln {((x \cdot 7-y))})^{(-1) \cdot z}+(-1) \cdot (\ln {((x \cdot 7-y))})^{(-1) \cdot z} \cdot ((0 \cdot z+(-1) \cdot 0) \cdot \ln {(\ln {((x \cdot 7-y))})}+(-1) \cdot z \cdot  \dfrac{1}{\ln {((x \cdot 7-y))}}  \cdot  \dfrac{1}{(x \cdot 7-y)}  \cdot ((0 \cdot 7+x \cdot 0)-1))))$

После очевиднейших упрощений, которые адекватный человек может сделать ещё в утробе, получаем:

$(\sin {(((-1) \cdot x+(y)^{z}))} \cdot (-1) \cdot z \cdot (y)^{(z-1)} \cdot (-1) \cdot (\ln {((x \cdot 7-y))})^{(-1) \cdot z}+((-789)+\cos {(((-1) \cdot x+(y)^{z}))}) \cdot (-1) \cdot (\ln {((x \cdot 7-y))})^{(-1) \cdot z} \cdot (-1) \cdot z \cdot  \dfrac{1}{\ln {((x \cdot 7-y))}}  \cdot  \dfrac{1}{(x \cdot 7-y)}  \cdot (-1))$

\newpageДифференцируем по z:

$g(z) = y$

Ежу понятно, что:

$g^{'}(z) = 0$

$g(z) = 7$

Петрович закопает того, кто не знает, что:

$g^{'}(z) = 0$

$g(z) = x$

Если приглядеться, то ты все равно не увидишь, что:

$g^{'}(z) = 0$

$g(z) = x \cdot 7$

Сложить 2 + 2 эквивалентно по сложности следующему:

$g^{'}(z) = (0 \cdot 7+x \cdot 0)$

$g(z) = (x \cdot 7-y)$

Петрович закопает того, кто не знает, что:

$g^{'}(z) = ((0 \cdot 7+x \cdot 0)-0)$

$g(z) = \ln {((x \cdot 7-y))}$

Ньютон перевернулся бы в гробу, если бы узнал, что ты не знаешь, что:

$g^{'}(z) =  \dfrac{1}{(x \cdot 7-y)}  \cdot ((0 \cdot 7+x \cdot 0)-0)$

$g(z) = \ln {(\ln {((x \cdot 7-y))})}$

Не понимаю тех, кто не знает, что:

$g^{'}(z) =  \dfrac{1}{\ln {((x \cdot 7-y))}}  \cdot  \dfrac{1}{(x \cdot 7-y)}  \cdot ((0 \cdot 7+x \cdot 0)-0)$

$g(z) = z$

Если бы меня разбудили в ночь после посвята, то я бы сходу ответил, что:

$g^{'}(z) = 1$

$g(z) = (-1)$

Ежу понятно, что:

$g^{'}(z) = 0$

$g(z) = (-1) \cdot z$

Не понимаю тех, кто не знает, что:

$g^{'}(z) = (0 \cdot z+(-1) \cdot 1)$

$g(z) = (-1) \cdot z \cdot \ln {(\ln {((x \cdot 7-y))})}$

Если бы меня разбудили в ночь после посвята, то я бы сходу ответил, что:

$g^{'}(z) = ((0 \cdot z+(-1) \cdot 1) \cdot \ln {(\ln {((x \cdot 7-y))})}+(-1) \cdot z \cdot  \dfrac{1}{\ln {((x \cdot 7-y))}}  \cdot  \dfrac{1}{(x \cdot 7-y)}  \cdot ((0 \cdot 7+x \cdot 0)-0))$

$g(z) = (\ln {((x \cdot 7-y))})^{(-1) \cdot z}$

Я сам выпал на этом моменте:

$g^{'}(z) = (\ln {((x \cdot 7-y))})^{(-1) \cdot z} \cdot ((0 \cdot z+(-1) \cdot 1) \cdot \ln {(\ln {((x \cdot 7-y))})}+(-1) \cdot z \cdot  \dfrac{1}{\ln {((x \cdot 7-y))}}  \cdot  \dfrac{1}{(x \cdot 7-y)}  \cdot ((0 \cdot 7+x \cdot 0)-0))$

$g(z) = (-1)$

Я сам выпал на этом моменте:

$g^{'}(z) = 0$

$g(z) = (-1) \cdot (\ln {((x \cdot 7-y))})^{(-1) \cdot z}$

Ежу понятно, что:

$g^{'}(z) = (0 \cdot (\ln {((x \cdot 7-y))})^{(-1) \cdot z}+(-1) \cdot (\ln {((x \cdot 7-y))})^{(-1) \cdot z} \cdot ((0 \cdot z+(-1) \cdot 1) \cdot \ln {(\ln {((x \cdot 7-y))})}+(-1) \cdot z \cdot  \dfrac{1}{\ln {((x \cdot 7-y))}}  \cdot  \dfrac{1}{(x \cdot 7-y)}  \cdot ((0 \cdot 7+x \cdot 0)-0)))$

$g(z) = z$

Сложить 2 + 2 эквивалентно по сложности следующему:

$g^{'}(z) = 1$

$g(z) = (y)^{z}$

В садике мне рассказывали, что:

$g^{'}(z) = \ln {(y)} \cdot (y)^{z} \cdot 1$

$g(z) = x$

Максимально тривиально, что:

$g^{'}(z) = 0$

$g(z) = (-1)$

Петрович закопает того, кто не знает, что:

$g^{'}(z) = 0$

$g(z) = (-1) \cdot x$

Я сам выпал на этом моменте:

$g^{'}(z) = (0 \cdot x+(-1) \cdot 0)$

$g(z) = ((-1) \cdot x+(y)^{z})$

Сложить 2 + 2 эквивалентно по сложности следующему:

$g^{'}(z) = ((0 \cdot x+(-1) \cdot 0)+\ln {(y)} \cdot (y)^{z} \cdot 1)$

$g(z) = \cos {(((-1) \cdot x+(y)^{z}))}$

Я сам выпал на этом моменте:

$g^{'}(z) = \sin {(((-1) \cdot x+(y)^{z}))} \cdot (-1) \cdot ((0 \cdot x+(-1) \cdot 0)+\ln {(y)} \cdot (y)^{z} \cdot 1)$

$g(z) = (-789)$

Я сам выпал на этом моменте:

$g^{'}(z) = 0$

$g(z) = ((-789)+\cos {(((-1) \cdot x+(y)^{z}))})$

Сложить 2 + 2 эквивалентно по сложности следующему:

$g^{'}(z) = (0+\sin {(((-1) \cdot x+(y)^{z}))} \cdot (-1) \cdot ((0 \cdot x+(-1) \cdot 0)+\ln {(y)} \cdot (y)^{z} \cdot 1))$

$g(z) = ((-789)+\cos {(((-1) \cdot x+(y)^{z}))}) \cdot (-1) \cdot (\ln {((x \cdot 7-y))})^{(-1) \cdot z}$

Если бы моя собака умела говорить, то сказала бы, что:

$g^{'}(z) = ((0+\sin {(((-1) \cdot x+(y)^{z}))} \cdot (-1) \cdot ((0 \cdot x+(-1) \cdot 0)+\ln {(y)} \cdot (y)^{z} \cdot 1)) \cdot (-1) \cdot (\ln {((x \cdot 7-y))})^{(-1) \cdot z}+((-789)+\cos {(((-1) \cdot x+(y)^{z}))}) \cdot (0 \cdot (\ln {((x \cdot 7-y))})^{(-1) \cdot z}+(-1) \cdot (\ln {((x \cdot 7-y))})^{(-1) \cdot z} \cdot ((0 \cdot z+(-1) \cdot 1) \cdot \ln {(\ln {((x \cdot 7-y))})}+(-1) \cdot z \cdot  \dfrac{1}{\ln {((x \cdot 7-y))}}  \cdot  \dfrac{1}{(x \cdot 7-y)}  \cdot ((0 \cdot 7+x \cdot 0)-0))))$

После очевиднейших упрощений, которые адекватный человек может сделать ещё в утробе, получаем:

$(\sin {(((-1) \cdot x+(y)^{z}))} \cdot (-1) \cdot \ln {(y)} \cdot (y)^{z} \cdot (-1) \cdot (\ln {((x \cdot 7-y))})^{(-1) \cdot z}+((-789)+\cos {(((-1) \cdot x+(y)^{z}))}) \cdot (-1) \cdot (\ln {((x \cdot 7-y))})^{(-1) \cdot z} \cdot (-1) \cdot \ln {(\ln {((x \cdot 7-y))})})$

\newpageТогда полный дифференциал будет:

$df = \dfrac{\partial f}{\partial x} \cdot dx + \dfrac{\partial f}{\partial y} \cdot dy + \dfrac{\partial f}{\partial z} \cdot dz = (\sin {(((-1) \cdot x+(y)^{z}))} \cdot (-1) \cdot (-1) \cdot (-1) \cdot (\ln {((x \cdot 7-y))})^{(-1) \cdot z}+((-789)+\cos {(((-1) \cdot x+(y)^{z}))}) \cdot (-1) \cdot (\ln {((x \cdot 7-y))})^{(-1) \cdot z} \cdot (-1) \cdot z \cdot  \dfrac{1}{\ln {((x \cdot 7-y))}}  \cdot  \dfrac{1}{(x \cdot 7-y)}  \cdot 7) \cdot dx + (\sin {(((-1) \cdot x+(y)^{z}))} \cdot (-1) \cdot z \cdot (y)^{(z-1)} \cdot (-1) \cdot (\ln {((x \cdot 7-y))})^{(-1) \cdot z}+((-789)+\cos {(((-1) \cdot x+(y)^{z}))}) \cdot (-1) \cdot (\ln {((x \cdot 7-y))})^{(-1) \cdot z} \cdot (-1) \cdot z \cdot  \dfrac{1}{\ln {((x \cdot 7-y))}}  \cdot  \dfrac{1}{(x \cdot 7-y)}  \cdot (-1)) \cdot dy + (\sin {(((-1) \cdot x+(y)^{z}))} \cdot (-1) \cdot \ln {(y)} \cdot (y)^{z} \cdot (-1) \cdot (\ln {((x \cdot 7-y))})^{(-1) \cdot z}+((-789)+\cos {(((-1) \cdot x+(y)^{z}))}) \cdot (-1) \cdot (\ln {((x \cdot 7-y))})^{(-1) \cdot z} \cdot (-1) \cdot \ln {(\ln {((x \cdot 7-y))})}) \cdot dz$

\end{flushleft}
\end{document}